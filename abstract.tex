Computational sensing has demonstrated the ability to ameliorate or eliminate many trade-offs in traditional sensors. Rather than attempting to form a perfect image, then sampling at the Nyquist rate, and reconstructing the signal of interest prior to post-processing, the computational sensor attempts to utilize a priori knowledge, active or passive coding of the signal-of-interest combined with a variety of algorithms to overcome the trade-offs or to improve various task-specific metrics. While it is a powerful approach to radically new sensor architectures, published research tends to focus on architecture concepts and positive results. Little attention is given towards the practical issues when faced with implementing computational sensing prototypes.

I will discuss the various practical challenges that I encountered while developing three separate applications of computational sensors. The first is a compressive sensing based object tracking camera, the \acrfull{scout}, which exploits the sparsity of motion between consecutive frames while using no moving parts to create a shift variant point-spread function. The second is a spectral imaging camera, the \acrfull{afssi-c}, which uses a modified version of Principal Component Analysis with a Bayesian strategy to adaptively design spectral filters for direct spectral classification using a digital micro-mirror device (DMD) based architecture. The third demonstrates two separate architectures to perform spectral unmixing by using an adaptive algorithm or a hybrid techniques of using Maximum Noise Fraction and random filter selection from a liquid crystal on silicon based computational spectral imager, the \acrfull{lcsi}. 

All of these applications demonstrate a variety of challenges that have been addressed or continue to challenge the computational sensing community. One issue is calibration, since many computational sensors require an inversion step and in the case of compressive sensing, lack of redundancy in the measurement data. Another issue is over multiplexing, as more light is collected per sample, the finite amount of dynamic range and quantization resolution can begin to degrade the recovery of the relevant information. A priori knowledge of the sparsity and or other statistics of the signal or noise is often used by computational sensors to outperform their isomorphic counterparts. This is demonstrate in all three of the sensors I have developed. These challenges and others will be discussed using a case-study approach through these three applications.



	

