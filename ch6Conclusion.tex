\chapter{Conclusion}\label{chap:Conclusion}

This dissertation discussed several important practical considerations of computational sensing and how they are addressed in three separate applications: compressive object tracking, adaptive spectral image classification, and compressive spectral unmixing. As computational sensing continues to make rapid progress in the coming years many of these issues must be addressed and confronted. 

The first chapter introduced the reader to the concepts of isomorphic and computational sensing. Computational sensing was developed from the disparate fields of multiplexing spectroscopy and indirect imaging. However, the development of the \gls{ccd}, the digital computer, and data compression allowed scientists to begin creating practical and reliable demonstrations of computational sensing. The advent of the mathematics and algorithms of compressive sensing generated an additional technological leap which dramatically reduced the number of measurements in order to reconstruct the signal of interest. Once the history and benefits of computational sensing was discussed, I then discussed the practical issues of computational sensors: calibration, over multiplexing due to finite dynamic range and quantization resolution, code design and optimization, and the need for a priori knowledge.

The second chapter formally developed the concepts related to computational sensing. I discussed the Fellgett advantage, using Hadamard and S-matrix multiplexing. Then a formal discussion of \acrfull{pca} was developed, which demonstrates how it can be used as a dimensionality reduction technique which creates a basis that decorrelates the data. Then a formal discussion on Bayesian statistics was given which justified their use in the \gls{afssi-c}. Sparsity, incoherence, and the \acrfull{rip} were discussed in-depth to demonstrate why random coding is popular for acquiring sparse signals in compressive sensing. The $\ell_1$ minimization subject to data agreement constraints and its equivalent problem minimizing the $\ell_1$ regularized least squares objective function were also discussed with some intuition as to why it works well for promoting sparse solutions to undetermined inverse problems. 

The \acrfull{scout} was discussed in the third chapter. The \gls{scout} showed how intentionally blurring the \acrfull{psf} in a compressive imager can actually help reduce the effects of overmultiplexing. As the amount of signals being sensed increases the variations in the sensed single become more difficult to discern. Overmultiplexing occurs due to the finite dynamic range and quantization resolution. In the single pixel camera \cite{duarte2008single}, one is forced to measure a series of projections in a time sequential manner prior to solving an inverse problem, in the \gls{scout} the projections occur in parallel for a single difference scene. The \gls{scout} introduced the reader to the importance of calibration in a compressive sensor. The measurement matrix must be carefully measured to produce optimal results. It also demonstrated that optimizing the measurement matrix can be extremely time consuming by hand. Developing a heuristic method based on the coherence metric and ray based simulations reduced the time it took to find optimal design parameters. 

The fourth chapter discussed the \acrfull{afssi-c} which is a computational spectral image sensor designed to adaptively classify spectra at each location in the field-of-view of the sensor. It used back-to-back spectrographs with a digital micro-mirror device (DMD) to produces psuedo-arbitrary spectral filters at each system pixel. After each measurement, an algorithm updates the probabilities of each spectra and weighted the spectral library before recomputing principal components. The first principal component vector is then used as the spectral filter for each measurement. The \gls{afssi-c} is shown to quickly converge to the correct spectrum significantly outperforming traditional spectral imaging architectures especially in low \gls{tsnr} scenarios. The chapter discussed the development of spatial and spectral calibration procedures, vital for optimal performance of the \gls{afssi-c}. A procedure for estimating the system noise was also described in-depth. 

The fifth chapter discussed an extension of spectral classification called spectral unmixing. Two architectures were discussed for spectral unmixing: the \acrfull{afss} and the \acrfull{lcsi}. The \gls{lcsi} relies on the wavelength dependent nature of birefringence to generate multiplex spectral measurements. One of the major constraints of the \gls{lcsi} is that psuedo-arbitrary spectral filters are not possible like in the \gls{afss} or \gls{afssi-c}, therefore I had to develop techniques for selecting spectral filters using random selections and a hybrid of using \gls{pca} or \gls{mnf} to select spectral filters and then using the random filter selections. 

\section{Future Outlook}


One research project that has been an active area of research in the \acrfull{LENS} is the coded memory effect imaging system being developed by my colleague Xiaohan Li. The project combines the memory effect of speckle with \acrfull{cacti} to temporally code speckle. This allows one to infer dynamic object information that is not directly observable with the speckle alone. 

Computational sensing continues to have an extremely promising future. Large companies are actively conducting research and developing products which take advantage of the principles of computational sensing. For example, Microsoft sells a depth sensor, called the Kinect, which relies on a form computational sensing to infer the distance of objects and their shapes. Meanwhile, a recently announced grant from NASA will fund the development of snapshot hyperspectral imagers based on the \acrfull{ctis} for space-borne applications \cite{ricespectraleyes2016grant}. Many university based research groups around the world are actively conducting research that demonstrate compact spatial, spectral, temporal, and polarization computational sensing. Computational sensing will become even more prevalent as the demand for higher resolutions in resource constrained environments abound. 
%\bibliographystyle{IEEEtranS}  
%\bibliography{ThesisBib}



