Graduate school is an arduous and enlightening experience. It is not difficult by design but by nature it forces one into a state of mind which embraces the edge of knowledge and trek into the unknown. I was fortunate to have many guides who showed me the path, even when there were times when I wandered off to get my bearings. Along the way I encountered many people who not only helped me with the journey but bestowed kindness and friendship, asking for nothing in return.

My main guide along the journey was Professor Michael Gehm. I first met him when I took a graduate level Linear Algebra course which I found particularly challenging. I often went to his office hours asking for help and his ability to be patient and explain concepts from different perspectives is a gift few teachers have. As an advisor, I would like to thank him for all of the help and guidance he has given me over the years. His generosity for funding my graduate studies as well trips to conferences is appreciated. He believed in me more than I believed in myself. I consider him not only as a mentor but as a father figure. If I can become half the scientist that he is, I would consider that a successful career.

I especially want to thank Professor Esteban Vera, who I first met as a postdoctoral researcher in the Laboratory for Engineering Non-Traditional Sensors (LENS) and supervised me for the majority of my graduate studies. Much of the work and results in this disseration is due to his guidance. Even after he started his professorship in Chile, he was willing to review my data and suggest different methods of analysis. Professor Vera is directly responsible for much of my training as an experimentalist. I consider Professor Vera as an older brother who was always there to protect me from the pitfalls of the graduate school journey. 

I also thank Doctor Dathon Golish. His approach to work and life was a calming effect in often stressful times. He made major contributions to the Adaptive Feature Specific Spectral Imaging-Classifier (AFSSI-C) and provided valuable feedback on various research projects and conference presentations.

Thank you Professor Mark Neifeld and Professor Amit Ashok for being my advisor and supervisor during my first year as a PhD student. They were the first to introduce me to many of the techniques and subjects related to computational sensing. They taught me fundamental concepts in optics, statistical signal processing and programming. Many of the results in this dissertation would not have been possible without their teachings.

I've also had many other supervisors along the way whose effort must also be acknowledged: My undergraduate advisor at San Diego State University, Professor Matthew Anderson. Doctor John Crane, who was my supervisor during my internship at the Lawrence Livermore National Laboratory. Professor Joseph Eberly and Professor Gary Wicks who were my advisors at the Institute of Optics. 

I would like to formally express gratitude to a number of exceptional teachers throughout my life. Professor Tom Milster who taught Diffraction and Interference and allowed me to be a teaching assistant for that course. Professor Masud Mansuripur, whose course in Electromagnetic Waves was the most elegant and well taught version of the classical nature of light that I have ever had the pleasure to experience. Professor Jeff Davis, who first ignited my passion for optics while I was an undergraduate physics student at San Diego State University.

I also want to thank several faculty members who committed time from their busy schedules to help with several milestones of my graduate school experience. Special thanks to Professor Julie Bentley, Doctor James Oliver, and Professor Richard Morris who wrote letters of recommendation for me. Appreciation goes to Professor Tom Milster, Professor Harrison Barrett, Professor Russell Chipman, and Professor John Greivenkamp who formed my oral comprehensive exam committee. Thank you to Professor Rongguang Liang who served on my doctoral dissertation committee.

I've had the good fortune to be exposed to amazing groupmates as part of the LENS. David Coccarelli invited our family to spend our first Thanksgiving in North Carolina with him and we had many discussions about college basketball and life. Matthew Dunlop-Gray designed and constructed the AFSSI-C which is the foundation for much the work in this disseration. Tariq Osman constructed the Static Computational Optical Undersampled Tracker (SCOUT) which is also a major part of this dissertation. Alyssa Jenkins whose combination of sense of humor and intelligence is unmatched. Thank you to Qian Gong, Kevin Kelly, Adriana DeRoos, David Landry, Xiaohan Li, and Dineshbabu Dinakarababu for your friendship. Finally, I consider Wei-Ren Ng as one of my best friends and as a brother. Our time in the LENS group was marked by many late nights spent working in the lab and office. He was generous in sharing his knowledge and gave me the advice that I often did not want to hear but was true. 

I would like to thank several members of the Duke Imaging and Spectroscopy Program (DISP) laboratory for their friendship: Patrick Llull, Mehadi Hassan, and Evan Chen. Tsung Han Tsai was not only a colleague but his work on computational polarimetry and spectroscopy using an Liquid Crystal on Silicon (LCOS) Spatial Light Modulator (SLM) was the inspiration which directly lead my idea of using the same architecture for computational spectral unmixing. 

Other graduate students, colleagues, and faculty must also be thanked, for at one time or another they all helped me: Basel Salahieh, Vicha Treeaporn, John Hughes, Steve Feller, Myungjun Lee, Sarmad H. Albanna, Professor Lars Furenlid, Doctor Joseph Dagher, Professor Daniel Marks, Professor Janick Roland-Thompson, Mary Pope, Mark Rodriguez, and Amanda Ferris. 

Appreciation goes to the all the staff and faculty at the College of Optical Sciences at the University of Arizona. It is one of the most friendly and well run academic departments I have ever had the fortune to be a part of. I hope my career will reflect well upon the college. 

Finally, I would like to thank my closest friends that I've met throughout the years. They often provided much needed respites during my journey---Christopher MacGahan, Ricky Gibson, Krista MacGahan, Kristi Behnke, Michael Gehl, Carlos Montances, Matthew Reaves, Vijay Parachuru, Eric Vasquez. Thank you for letting me into your lives and being part of mine.

Last but not least, to my family. Thank you. 