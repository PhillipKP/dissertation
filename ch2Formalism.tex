\chapter{Formalism}\label{chap:Formalism}

This chapter introduces the reader to the more rigorous concepts and mathematical background that will required to fully understand the material presented in the later chapters of this dissertation. 

A rigorous discussion of multiplexing and signal-to-noise ratio will be discussed, as well are various coding schemes used in various notable computational sensors as well as the ones in this disseration. 

Since the Adaptive Feature Specific Spectral Imaging-Classifier (AFSSI-C) relies on a variation of Principal Component Analysis (PCA) and a Bayesian algorithm for coding design we will discuss some of the fundamental of Bayesian probability and the Log-Likelihood Ratios. 



\section{Multiplexing}

\section{Principal Component Analysis}

\section{Bayesian Rules and Log-Likelihood Ratios}

\section{Compressive Sensing}

\subsection{The Nyquist-Shannon Sampling Theorem}

The Nyquist-Shannon Sampling Theorem states one must sample a signal with a sampling rate that is at least twice the maximum frequency of the signal to prevent aliasing \cite{shannon1949communication}.


\subsection{Sparsity, Incoherence, and the Restricted Isometry Property}

\subsection{Inversion}

\subsubsection{L0 and L1 Norm Minimization}

\subsubsection{LASSO and sparsity regularization}

%\bibliographystyle{IEEEtranS}  
%\bibliography{ThesisBib}



