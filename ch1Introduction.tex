\chapter{Introduction}\label{sec:chIntrosecHistory}

This chapter introduces the reader to the concept of computational sensing and provides the groundwork for why there is a need to address the practical issues in experimental computational sensing. Rather than a rigorous discussion, this chapter will focus on a big picture view of computational sensing and provide intuitive understand to the reader. 

The chapter is designed to provide a conversational discussion of the field and the main contribution of this disseration. A rigorous discussion of the concepts is given in \autoref{chap:Formalism}.


%%%%%%%%%%%%%%%%%%%%%%%%%%%%%%%%%%%%%%%%%%%%%%
%%%%%%%%%%%%%%%%%%%%%%%History%%%%%%%%%%%%%%%%%%%%

\section{A Historical Development of Computational Sensing}\label{sec:chIntrosecHistory}

As field, computational sensing was not formed from a single seminal paper or discovery but rather a series of disparete 

It can be argued that computational sensing began around World War II. Computational sensing is at least several decades old. Argubly the first 

%This section explores a brief history of the development of nuclear weapons---from their development and first use in 1945 to the buildup of the arms race (\cref{sec:chIntrosecHistorysubsecBuildup}). This culminated in the Cuban Missile Crisis (\cref{sec:chIntrosecHistorysubsecCMC}), which is considered the height of the Cold War. A summary is presented of important treaties signed between the United States (U.S.), Soviet Union (U.S.S.R.) and other nuclear and non-nuclear states over the past several decades in~\cref{sec:chIntrosecHistorysubsecNPT}. To verify compliance with future treaties, new technologies will be needed. The challenges associated with these different problems are summarized in~\cref{sec:chIntrosecHistorysubsecFuture}. More details, and summaries of interesting historical events, can be found in Charles Loeber's book~\citep{IntroBuildingtheBomb}. 




%\bibliographystyle{IEEEtranS}  
%\bibliography{ThesisBib}

