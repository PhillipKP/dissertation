\chapter{Computational Spectral Unmixing}\label{chap:Csu}

\section{Motivation}

As I discussed in \Cref{chap:Afssic}, spectral imaging is an important sensing technique in wide a variety of areas. I made the argument that most of the time, material classification based on their spectra is the ultimate goal of spectral imaging. Since the \gls{afssi-c} was a proof-of-principle experimental prototype, each spectra in the spectral source we used was a pure spectum. In other words, we displayed one spectrum from the library at each spatial pixel location. 

There are certain situations, especially in remote sensing, one or more materials maybe present at each spatial pixel. In this case, the spectrum that is measured is a combination of constituent spectra. We call the consituent spectra \glspl{endmember} and the combined spectrum the \gls{mixed spectrum}.

If one is interested in quantify how much each material is present in the pi el then spectral classification is not the preferred task. Spectral classification alone may only provide data for one of the spectra and ignore the contribution of the other spectra. At worse, the presence of other materials besides the dominant material could degrade classification results. 

Spectral unmixing is a task which quantifies how much of each \glspl{endmember} is present in the \gls{mixed spectrum}. In the field of spectral unmixing, the spectral library is also known as the endmember library, which is simply the set of possible spectra that one is concerned with discerning \cite{lillesand2014remote}. The amount of each of the consitutent spectra is quantified by their respective \gls{fractional abundance}.

When spectral imaging is used in remote sensing, mixed pixels occur because of two distinct reasons \cite{keshava2002spectral, keshava2003survey}. First, if the spatial resolution of the sensor is low enough that seperate materials can jointly occupy the \acrfull{fov} of a single pixel, the resulting spectral measurement is a combination of the spectra. This occurs when the distance from the object scene to the sensor produces such low magnificiations that a single pixel may be limited in resolving a several meters [CITE]. The second reason for mixed pixels occurs when different materials are combined into a approximately homogenous mixture. In this case, mixed spectrum is not due to the lack of spatial resolution but due to the inherent nature of the object being imaged.

For the purposes of this chapter I will focus on the first case where mixed spectra arise due to the spatial resolution limitation of the sensor. In this case, we can imagine the object scene as a \emph{checkerboard} mixture: light from the illumination source scatters or reflects from only one of the materials before being observed by the sensor. Therefore we ignore multiple scattering between materials. In this case the fractional abundance of the measured spectra and the area of each relative area of each material incident onto the pixel show a linear relationship which is called the \gls{lmm}. In the \gls{lmm}, the interactions amoung distinct endmembers are assumed to be neglible \cite{clark1984reflectance}. The mixed spectrum can be written as
%
\begin{equation}
\mb{f} = \sum_{r=1}^{N_{R}} x_r s_r + \mb{e} = \mb{S} \mb{x}  + \mb{e}
\end{equation}
%
where $N_R$ is the number of endmembers in the spectral library. Each spectra has $N_{\lambda}$ spectral channels. $\mb{x}$ is the \gls{fractional abundance} an $N_{\lambda}$ fractional vector. Physically we should two several constraints on theat the \gls{lmm}. The \emph{nonnegativity} constraint requires all the fractional abundances are nonnegative, 
%
\begin{equation}
	x_r \geq 0
\end{equation}
%
and the \emph{additivty} constaint requires that the fractional abundance vector sums to one
%
\begin{equation}
	\sum_{r = 1}^{N_{\lambda}} x_r = 1
\end{equation}


In isomorphic spectral imaging, the spectral datacube is first acquired by the instrument before any unmixing step is performed. Because of this a large amount of data is generated by measuring the entire spectral datacube. In traditional spectral imaging, significant attention has been focused on the computationa burder of hypspectral processing induces by the high dimensionality of the data. Prior to the unmixing step, the spectral datacube is first processing using a \gls{dimension reduction} step which is used to significantly reduce the computational load of the unmixing step. In cases where the spectral library or endmembers are not known before hand, an \gls{endmember determination} step is useds to estimate the spectra that constiture the mixed spectra, and finally the \gls{inversion} step is used to solve the unmixing problem of determining the \gls{fractional abundance} of each mixed pixel in the spectral datacube. For the purposes of this chapter I will assume that the spectral library is known, however I readily ackowledge scenarios in which endmember detection should also be incorporated. 


In this chapter, I will discuss my efforts to apply the techniques of computational sensing to spectral unmixing. Computational sensing techiniques have the potential to greatly improve the performance of spectral unmixing by incorporating the dimension reduction step as part of the measurement process. By utilizing the Fellgett and Jacquinot advantage, we can signficiantly improve unmixing performance by combining the physical step of measurement and the processing step of dimension reduction. Furthermore, I will discuss I will discuss two seperate computational spectral imaging architectures which can be used to obtain performance gains on the spectral imager. Aside from invoking leveraging the Fellgett and Jacquinot advantage, one can also use the fact that in a mixed spectrum, only a few of all the possible endmembers are actually present. Thus one can invoke sparsity which allows one to reduce the solution space of the inverse problem.


Several researchers have already shown promising results in applying compressive sensing to spectral unmixing using a modified single-pixel camera architecture \cite{li2012compressive}. They demonstrated the ability to reconstruct the fractional abundance planes without the need to explicitly reconstruct the spectral datacube. In this approach, the object scene is imaged onto a \gls{dmd} and then a condensor lens focuses the reflected light into a whiskbroom spectrometer. One can think of this architecture parallel single-pixel cameras each operating at a different spectral channel, with the constraint that each \gls{dmd} must display the same pattern. This architecture does not code the spectral dimension of the spectral datacube. The researchers demonstrated compressive unmixing by minimizing the total variation (TV) of the endmember images. For example, in a normal computer monitor we typically have three endmembers images: the red image, the green images, and the blue image, which are combined to form the full color image. Thus the researched minimizing the total variation of the image while enforcing the nonnegativity constraint and the addivity constraint. 

In another effort, researchers use the \acrfull{cassi} architecture to perform compressive sensing on the spectral datacube and solve the $\ell_1$-regularized least squares problem to promote sparsity in the fractional abundances. Due to the nature of the single-disperser \gls{cassi} architecture, the researchs are forced to solve a larger joint-inference problem to preform spectral unmixing.

In my research I propose the use of two architectures that has not yet been explored for computational spectral unmixing both of which coded the spectral datacube using spectral filters. The first architecture, the reader is already familure which from \Cref{chap:Afssic} is the \acrfull{afssi-c}. The second architecture, is a \acrfull{lcos} based spectral imager which allows for an extremely compact computational spectral imager. I will present simulation and initial experimental results using the \gls{afssi-c} architecture and promising simulation results from the \gls{lcos} based spectral imager and discuss a filter selection technique that improves unmixing performance.

%\bibliographystyle{IEEEtranS}  
%\bibliography{ThesisBib}

\section{Forward Model}



In the \gls{tunable filter}, the forward model is 

The linear mixing model. A mixed spectrum $\mb{f}$

\begin{equation}
\mb{f} = \sum_{r=1}^{N_{R}} x_r s_r + \mb{e} = \mb{S} \mb{x}  + \mb{e}
\end{equation}

\section{Inverse Algorithms}

Inverse is the step of estimating the fractional abundance vector $\mb{x}$. The majority of inversion algorithms for spectral unmixing attempt to minimize some objective function related to squared-error [CITE ]algorithms