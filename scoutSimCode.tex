\chapter{SCOUT Simulation Code}\label{app:scoutSimCode}

\begin{lstlisting}[
  style=Matlab-editor,
  basicstyle=\tiny,
  escapechar=`,
  caption={Main Simulation Script},
]
% GENERATING SYSTEM MEASUREMENT MATRIX 'H'

clc; clearvars; close all
rand_trials = 5
% lens to camera image sensor defocus in mm
defocus_value=35;
% pitch (in mm)of mask1 (next to image sensor)
p1=0.01; 
% pitch (in mm) of mask2 (next to lens)
p2=0.5;
%focal length of lens
f=35; 

%% comments about image sensor pitch
%  reading 288 x 288 part on image sensor and binning 
%  (36x36) to obtain 8x8 pixels
%  sensor pitch is 9 microns. For binnned 8x8, each pixel
%  is 36*9=   324 microns

% pitch of image sensor in mm
p=0.324;

for i = 1:rand_trials
    
    % d1 is distance between Mask 1 and image plane of lens
    % here 2.04 mm is the spacing between camera sensor
    % image plane and Mask 1
    d1=defocus_value-2.04;
    
    % d2 is distance between Mask 2 and image plane of lens
    % here 6.35 mm is the spacing between
    d2=f-6.35;
    
    % calculating pitch and positions relative to detector
    % for locating Mask1 away from focus
    pos1 = (defocus_value -d1)/(f+defocus_value);
    
    % position of Mask 2
    pos2=(d2+defocus_value)/(f+defocus_value); 
    
    %calculating pitch values of Masks on detector
    pd1=(p1*defocus_value)/d1; % for mask1
    pd2=(p2*defocus_value)/d2; % for mask2
    pitch1=pd1/p;
    pitch2=pd2/p;
    
    
    pitch = [pitch1 pitch2];
    pos=[pos1 pos2];
    
    % Mask leaakge based on experimental mask transmission 
    % values
    leakage1=0.12;%masks open part leakage of 12%
    leakage=0.00;%masks black part no leakage
    
    % fill factor of mask1 and mask2 , 50%
    through=[0.5 0.5]; 
    
    lens_defocus=defocus_value;
    
    %scaling mask patterns, using smaller values speeds up code
    upsamp=0.25; 
    
    directory=pwd;
    
    date='feb16'; %date for file names
    
    name = ...
        sprintf(['%s/pos_%.3g_%.3g_through_%.3g_%.3g_'...
            'pitch_%.3g_%.3g.mat'],...
        directory,pos(1),pos(2),through(1),through(2),...
        pitch(1),pitch(2));
    
    % choose blur type on masks
    psfblur='fourierpsf';
    %psfblur='false' for no blur    
    % for adding blur based on fourier filter    
    %psfblur='fourierpsf' 
    
    varargin={'upsample',upsamp,'mask_pos',...
        pos,'throughput',through,'maskpitch', ...
        pitch,'defocus', lens_defocus,'psfblur',...
        psfblur,'leakage',leakage,'leakage1',leakage1};
    
    N=32;%size of the scene (NxN)
    K=8;%size of detector (KxK)
    
    % call function to generate system matrix, with point 
    % by point calibration. Selecting display as 'false' 
    % to prevent figure display while generating matrix
    H(:,:,i) = struct_projection(N, 'display',false,...
        'compression', N/K, varargin{:});
    

    save(['condnum_df_' num2str(defocus_value) ...
        'normalization_on.mat'],'H','rand_trials')
    
end

    % use the statement below for locating Mask1 before the focus (i.e. on the same side of focus as Mask 2)
    %  pos1=(d1+defocus_value)/(f+defocus_value);
    
    pos2=(d2+defocus_value)/(f+defocus_value); % position of Mask 2
    
    %calculating pitch values of Masks on detector
    pd1=(p1*defocus_value)/d1; % for mask1
    pd2=(p2*defocus_value)/d2; % for mask2
    pitch1=pd1/p;
    pitch2=pd2/p;
    
    
    pitch = [pitch1 pitch2];
    pos=[pos1 pos2];
    
    %Mask leaakge based on experimental mask transmission values
    leakage1=0.12;%masks open part leakage of 12%
    leakage=0.00;%masks black part no leakage
    
    
    through=[0.5 0.5]; % fill factor of mask1 and mask2 , 50%
    lens_defocus=defocus_value;
    upsamp=0.25; %scaling mask patterns, using smaller values speeds up code
    directory=pwd;
    date='feb16'; %date for file names
    
    name = sprintf('%s/pos_%.3g_%.3g_through_%.3g_%.3g_pitch_%.3g_%.3g.mat',directory,pos(1),pos(2),through(1),through(2),pitch(1),pitch(2));
    
    %%choose blur type on masks
    psfblur='fourierpsf';
    %psfblur='false' for no blur
    %psfblur='fourierpsf' for adding blur based on fourier filter
    
    
    varargin={'upsample',upsamp,'mask_pos',pos,'throughput',through,'maskpitch', pitch,'defocus', lens_defocus,'psfblur',psfblur,'leakage',leakage,'leakage1',leakage1};
    
    N=32;%size of the scene (NxN)
    K=8;%size of detector (KxK)
    
    % call function to generate system matrix, with point by point calibration.
    %selecting display as 'false' to prevent figure display while generating matrix
    H(:,:,i) = struct_projection(N, 'display',false,...
        'compression', N/K, varargin{:});
    
    % system_mat_name=sprintf('%s_defocus_%d.mat',date,lens_defocus);% file
    % name to save system matrix 'H'
    
    %     system_mat_name = ['trial' num2str(i) '_default_matrix.mat']
    save(['condnum_df_' num2str(defocus_value) 'normalization_on.mat'],'H','rand_trials')
    
end

\end{lstlisting}


\begin{lstlisting}[
  style=Matlab-editor,
  basicstyle=\tiny,
  escapechar=`,
  caption={struct\_projection function},
]
function H = struct_projection(N,varargin)

defopt.mask_pos = [.1 .9];
defopt.compression = 4;
defopt.throughput = 0.5;
defopt.defocus = 2;
defopt.display = true;
defopt.upsample = 4;

% Sets pitch equal to reconstruction resolution
defopt.maskpitch = 1; 
defopt.psfblur='false';
defopt.leakage1=0;%mask clear part
defopt.leakage=0;%mask dark part

opt = matlab_options(varargin, defopt);

compression = opt.compression;
throughput = opt.throughput;
dd = opt.defocus;
leakage = opt.leakage;
leakage1 = opt.leakage1;

upsample = opt.upsample;  % linear upsample factor

display = opt.display;

psfblur=opt.psfblur; % read blur type

% all physical distances are in mm


N=32;% scene size NxN

% location pitch - desired image-space reconstruction 
% resolution in mm
p  = 0.324;  

f = 35; % lens focal length in mm
D=25.4; %lens diameter in mm

maskcount = length(opt.mask_pos);

%for locating Mask1 away from focus
d1 = dd - opt.mask_pos(1)*(f+dd); 

% use the statement below for locating Mask1 before the 
% focus (i.e. on the same side of focus as Mask 2)

%%% testing for defocus=1, mask1 behind mask2

if maskcount>1
  d2 = opt.mask_pos(2)*(f+dd) - dd; %%%edit
else
  d2 = 0;
end
if maskcount>2
  d3 = dd+opt.mask_pos(3)*(f+dd); %%edit
else
  d3 = 0;
end

if maskcount>3
  error('mask_pos is too long, must have length 1-3');
end

throughput = throughput(:);
if length(throughput)==1 && maskcount>1
  throughput=throughput^(1/maskcount)*ones(maskcount,1);
end
if length(throughput) <3
  throughput = [throughput;ones(3-length(throughput))];
end

if length(opt.maskpitch) == 1 && maskcount > 1
  maskpitch=opt.maskpitch*ones(maskcount,1);
else
  maskpitch = opt.maskpitch(:);
end

% Code assumes all masks present even if they aren't, 
% so pad to length 3
maskpitch = [maskpitch ; ones(3 - length(maskpitch),1)];

% make mask pitch (pd) equal to reconstruction pitch (p)
pd = maskpitch * p; % mask pitch in sensor plane

%% perform geometrical calculations

p1 = pd(1)*d1/dd; % physical mask pitch---- mm real mask
p2 = pd(2)*d2/dd;
p3 = pd(3)*d3/dd;

% this is the pitch of the points in the aperture,
% a larger defocus (dd) gives a smaller pitch
pL = p*f/dd;  

fprintf('mask 1: d1 = %.1f, p1 = %.3f\n', d1, p1);
fprintf('mask 2: d2 = %.1f, p2 = %.3f\n', d2, p2);
fprintf('mask 3: d3 = %.1f, p3 = %.3f\n', d3, p3);

Sf = p * (N-1) / (1 + dd/f);  % full "image" width at focus (mm)
Sd = Sf * (1 + dd/f); % image width at sensor (at z = dd)
S1 = Sf * (1 + d1/f); % at mask 1
S2 = Sf * (1 + d2/f); % at mask 2 

S3 = Sf * (1 - d3/f); % at mask 3

Bd = D * dd/f; % blur width at sensor
B1 = D * d1/f;
B2 = D * d2/f;
B3 = D * d3/f;

%% define mask structures
m0.z = f; % aperture
m0.x = [-D/2:pL:D/2]; 
m0.y = m0.x;
[X,Y] = meshgrid(m0.x, m0.y);

%%## adding gaussian blur to model lens defocus

% 'blursize' indicates the size of the blur filter which is
% computed as per the defocus level above(bigger blur for a larger defocus).
blursize=size(X);

% the standard deviation of 12 is chosen to model lens 
% defocus with Gaussian blur
m0.v=fspecial('gaussian',blursize(1),12); 

m1.z = d1; % first mask
m1.x = [-(B1+S1)/2:p1:(B1+S1)/2];
m1.y = m1.x;
m1.v = double(rand(length(m1.x)) <= throughput(1));

% including leakage of dark part
leak = leakage*ones(length(m1.v)); 

% including leakage of clear part
leak1 = leakage1*ones(length(m1.v)); 
m1.v = max(m1.v.*(1-leak1),leak);
m2.z = d2; % second mask
m2.x = [-(B2+S2)/2:p2:(B2+S2)/2];
m2.y = m2.x;
m2.v = double(rand(length(m2.x)) <= throughput(2));

% including leakage of dark part
leak = leakage*ones(length(m2.v));

% including leakage of clear part
leak1 = leakage1*ones(length(m2.v));
m2.v = max(m2.v.*(1-leak1),leak);
m3.z = d3; % third mask
m3.x = [-(B3+S3)/2:p3:(B3+S3)/2];
m3.y = m3.x;
m3.v = double(rand(length(m3.x)) <= throughput(3));
% including leakage of dark part
leak = leakage*ones(length(m3.v));
% including leakage of clear part
leak1 = leakage1*ones(length(m3.v));
m3.v = max(m3.v.*(1-leak1),leak);

%% perform calculations
H = [];Hcrop=[];
% range and sampling for masks in sensor plane
xx = [-Sd/2:pd/upsample:Sd/2]; 
% range of point locations (mapped to image space)
xlist = [-Sd/2:p:Sd/2]; 
progress(1, length(xlist)^2, 1, 1);
ind = 1;

countval=length(xlist);

for xcount=1:countval
    for ycount=1:countval
        x=xlist(xcount);
        y=xlist(ycount);
        progress(ind); ind = ind + 1;
        [mout, ml] = overlay_masks({m0, m1, m2}, ...
            [x,y], dd, xx, xx,B1,B2,Bd,psfblur,D);
        full_image = mout.v;
        scalefac = upsample * p / pd;
        high_res = imresize(full_image, [N,N], 'bilinear');
        % default compression of 4 for  32x32 scene to 
        % 8x8 detector              
        low_res = bin_image(high_res, compression); 
        
        if display
            if gcf ~= 2; figure(2); end
            subplot(1,2,1);
            imagesc(high_res); axis image; colormap gray;
            subplot(1,2,2);
            imagesc(low_res); axis image; colormap gray;
            drawnow;
        end
        
        H = [H low_res(:)]; % build the system matrix
       
    end
end
if display
  figure(3); 
  imagesc(H); 
  colormap jet; 
  colorbar;
end
\end{lstlisting}


\begin{lstlisting}[
  style=Matlab-editor,
  basicstyle=\tiny,
  escapechar=`,
  caption={overlay\_masks function},
]
function [m_out, ml] = overlay_masks(mask_list, origin, ...
    z, x, y,B1,B2,Bd,psfblur,D)
% project a list of masks to given z plane and overlay them
%   mask_list - list of mask structures
%   origin - [x, y] coordinates of projection origin
%   z      - z-plane to project to
%   x, y   - x and y coordinate lists to interpolate onto

if nargin < 4
    x = ml{1}.x;
    y = ml{1}.y;
end

% project the masks to common z plane
if nargin > 1
    for ind = 1:length(mask_list)
        ml{ind} = project_mask(mask_list{ind}, origin, z);
    end
else
    ml = mask_list;
end

% build the combined mask structure
m_out = ml{1};
m_out.x = x;
m_out.y = y;
[X,Y] = meshgrid(x,y); % gridded coordinates for output mask
m_out.v = ones(size(X));

for ind = 1:length(ml)
    [Xm, Ym] = meshgrid(ml{ind}.x, ml{ind}.y); % native projected coordinates
    if 0
        fprintf('ind = %i\n', ind);
        fprintf('size(Xm) = [%i, %i]\n', size(Xm,1), ...
            size(Xm,2));
        fprintf('size(Ym) = [%i, %i]\n', size(Ym,1), ...
            size(Ym,2));
        fprintf('size(v)  = [%i, %i]\n', ...
            size(ml{ind}.v,1), size(ml{ind}.v,2));
        fprintf('size(X)  = [%i, %i]\n', size(X,1), ...
            size(X,2));
        fprintf('size(Y)  = [%i, %i]\n', size(X,1), ...
            size(X,2));
    end
    % interpolate from the native projected coordinates
    % onto the common coordinates
    
    switch (psfblur)
        
        case 'fourierpsf' % adding blur on masks
            switch (ind)
                case 1
                    m_out.v = m_out.v .* interp2(Xm, Ym, ...
                        ml{ind}.v, X, Y, '*linear', 0);
                    
                    
                case 2 %blur on mask 1
                    [sz1,sz2]=size((interp2(Xm, Ym, ...
                        ml{ind}.v, X, Y, '*linear', 0)));
                    
                    %create filter in fourier domain
                    blur_radius1=(sz1*((D-B1)/D))/2; 
                    % compute blur radius at chosen defocus 
                    % level
                    [rr cc] = meshgrid(1:sz1);
                    blurpsf1 = fftshift(sqrt((rr-round(sz1/2)).^2+(cc-round(sz1/2)).^2)<=blur_radius1);
                    blur_filter1 = abs((ifft2(blurpsf1)));
                    %apply filter
                    
                    filtered_mask =  ...
						real(ifft2(blurpsf1.*fft2(interp2(Xm, Ym, ml{ind}.v, X, Y, '*linear', 0))));
                    
                    %normalized the filtered mask to it can't amplify
                    filtered_mask = filtered_mask / max(filtered_mask(:));
                    
                    m_out.v = m_out.v .* filtered_mask;
                    m_out.v(m_out.v<0)=0;
                    
                case 3 %blur on mask 2
                    
                    [sz1,sz2]=size((interp2(Xm, Ym, ml{ind}.v, X, Y, '*linear', 0)));
                    
                    %create filter in fourier domain
                    blur_radius2=(sz1*((D-B2)/D))/2; % compute blur radius at chosen defocus level
                    [rr cc] = meshgrid(1:sz1);
                    blurpsf2 = fftshift(sqrt((rr-round(sz1/2)).^2+(cc-round(sz1/2)).^2)<=blur_radius2);
                    %apply filter
                    blur_filter2 = abs(ifft2(blurpsf2));
                    
                    filtered_mask = ...
						real(ifft2(blurpsf2.*fft2(interp2(Xm, Ym, ml{ind}.v, X, Y, '*linear', 0))));
                    
                    filtered_mask = filtered_mask / max(filtered_mask(:));
                    m_out.v = m_out.v .* filtered_mask;
                    m_out.v(m_out.v<0)=0;
            end
            
        case 'false' %no blur
            m_out.v = m_out.v .* interp2(Xm, Ym, ml{ind}.v, X, Y, '*linear', 0);
            
    end
    
end

\end{lstlisting}


\begin{lstlisting}[
  style=Matlab-editor,
  basicstyle=\tiny,
  escapechar=`,
  caption={project\_mask function},
]

function m_out = project_mask(m_in, origin, new_z)
% project the input mask to a new z plane.  The projection is performed
% via rays emanating from the provided x-y origin in the z=0 plane

% mask attributes: z, v, x, y (optional)
% if y is provided, v should be 2d, otherwise 1d

m_out.z = new_z;
m_out.v = m_in.v;
x0 = origin(1);
m_out.x = (m_in.x - x0) * (m_out.z/m_in.z) + x0;
if length(origin) > 1
    y0 = origin(2);
    m_out.y = (m_in.y - y0) * (m_out.z/m_in.z) + y0;
end

%calling function to locate the center of the aperture at the point source
%co-ordinates

 % call function to center PSF 
[m_out.x m_out.y]=centering_atlocation(origin,m_out.x,m_out.y); 

end

\end{lstlisting}

\begin{lstlisting}[
  style=Matlab-editor,
  basicstyle=\tiny,
  escapechar=`,
  caption={centering\_atlocation function},
]
function [out1, out2]=centering_atlocation(origin,in1,in2)

%function to locate the center of the aperture at the point source co-ordinates

m_out.x=in1;
m_out.y=in2;

%####calculations for x-cordinate####
len_mx=length(m_out.x);

test_even_odd=mod(len_mx,2);
if test_even_odd==0
    %even length
    mid_pt=(len_mx/2);
elseif test_even_odd==1
    %odd length
    mid_pt=((len_mx+1)/2);
end
% calculating mid point
midval=m_out.x(mid_pt);
m_out.x=(m_out.x - midval +  origin(1));%center the data around origin


%####calculations for y-cordinate####
len_my=length(m_out.y);
test_even_odd=mod(len_my,2);
if test_even_odd==0
    %even length
    mid_pty=(len_my/2);
else
    %odd length
    mid_pty=((len_my+1)/2);
end
midvaly=m_out.y(mid_pty);
m_out.y=(m_out.y - midvaly +  origin(2));%center the data around origin

out1=m_out.x;
out2=m_out.y;
%@@@@@@@@@@@@
   

\end{lstlisting}


\begin{lstlisting}[
  style=Matlab-editor,
  basicstyle=\tiny,
  escapechar=`,
  caption={bin\_image function},
]
function imout = bin_image(im, n)
% bin images
% im - image (or set of images) to be binned
% n  - number of elements (linearly) to be binned
%
% n can be a vector of bin sizes in each dimension.  If n is a scalar,
% it is interpreted as 
%    [n, n]       for 2D images, 
%    [n, n, 1]    for 3D images,
%    [n, n, 1, 1] for 4D images, etc.
%
% examples:
%  size(im)         n        -->  size(imout)
%  [50, 50]         5        -->  [10, 10]
%  [50, 50, 3]      5        -->  [10, 10, 3]
%  [50, 50, 100]    [5,5,20] -->  [10, 10, 5]
%
% note: if the dimensions of im are not multiples of the elements of n,
% then the last few values of im will not be used.  That is,
%   size(im) = [51, 51]   and  n = 5   -->  size(imout) = [10, 10]
% and the last row and column of im [im(:,51) and im(51,:)] are unused.



% expand n into vector form if necessary
if length(n) == 1
    ntmp = n;
    n = ones(1, length(size(im)));
    n(1) = ntmp;
    n(2) = ntmp;
end

s = floor(size(im)./n); % target size
S.type = '()';  % used in subsref and subsasgn
subscell = {};  %
for d = 1:length(s)
    subscell{d} = ':';
end

for d = 1:length(s)  % loop over each dimension
    if n(d) == 1; % no binning along this dimension - move on
        continue
    end
    S.subs = subscell;
    ts = size(im);
    ts(d) = s(d);
    imout = zeros(ts);
    for i = 1:s(d)
        start = (i-1)*n(d)+1;
        stop  = i*n(d);
        S.subs{d} = [start:stop];
        M = subsref(im, S);
        S.subs{d} = i;
        imout = subsasgn(imout, S, sum(M, d));
    end
    if d < length(s) % only need to reassign if we're not done yet
        im = imout;
    end
end

\end{lstlisting}