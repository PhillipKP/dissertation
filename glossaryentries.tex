

%%%%%%%%%%%%%%%%%%%%%%%%%%% Define acronyms here% %%%%%%%%%%%%%%%%%%%%%%%%%%%%%

\newacronym{svm}{SVM}{support vector machine}
\newacronym[plural=FPAs,firstplural=focal-plane arrays (FPAs)]{fpa}{FPA}{focal-plane array}
\newacronym{psf}{PSF}{point-spread function}
\newacronym{adc}{ADC}{analog-to-digital converter}
\newacronym{afssi-c}{AFSSI-C}{Adaptive Feature Specific Spectral Imaging-Classifier}
\newacronym{pca}{PCA}{Principal Component Analysis}
\newacronym{ppca}{pPCA}{probabilistically weighted Principal Component Analysis}
\newacronym{LENS}{LENS}{Laboratory for Engineering Non-Traditional Sensors}
\newacronym{scout}{SCOUT}{Static Computational Optical Undersampled Tracker}
\newacronym{disp}{DISP}{Duke Imaging and Spectroscopy Program}
\newacronym{lcos}{LCOS}{Liquid Crystal on Silicon}
\newacronym{slm}{SLM}{Spatial Light Modulator}
\newacronym{snr}{SNR}{signal-to-noise ratio}
\newacronym{swap-c}{SWAP-C}{size, weight and power-cost}
\newacronym{ct}{CT}{Computed Tomography}
\newacronym{spect}{SPECT}{Single-Photon Emission Computed Tomography}
\newacronym{pet}{PET}{Positron Emission Tomography}
\newacronym{mri}{MRI}{Magnetic Resonance Imaging}
\newacronym{radar}{Radar}{RAdio Detection And Ranging}
\newacronym{sar}{SAR}{Synthetic Aperture Radar}
\newacronym{ccd}{CCD}{Charge-Coupled Device}
\newacronym{cmos}{CMOS}{Complementary Metal–Oxide–Semiconductor}
\newacronym{cassi}{CASSI}{Coded Aperture Snapshot Spectral Imaging}
\newacronym{dmd}{DMD}{Digital Micro-Mirror Display}
\newacronym{cosi}{COSI}{Computational Optical Sensing and Imaging}
\newacronym{mse}{MSE}{Mean Squared Error}
\newacronym{rmse}{RMSE}{Root Mean Squared Error}
\newacronym{fts}{FTS}{Fourier Transform Spectrometer}
\newacronym{map}{MAP}{Maximum A Posteriori}
\newacronym{rip}{RIP}{restricted isometry property}
\newacronym{ls}{LS}{least squares}
\newacronym{osa}{OSA}{Optical Society of America}
\newacronym{lar}{LAR}{Least Angle Regression}
\newacronym{afss}{AFSS}{Adaptive Feature Specific Spectrometer}
\newacronym{cacti}{CACTI}{Coded Aperture Compressive Temporal Imaging}
\newacronym{aviris}{AVIRIS}{Airborne Visible/Infrared Imaging Spectrometer}
\newacronym{ctis}{CTIS}{Computed Tomography Imaging Spectrometer}
\newacronym{roi}{ROI}{Region of Interest}
\newacronym{sp}{SP}{system pixel}
\newacronym{tsnr}{TSNR}{Task Signal-To-Noise Ratio}
\newacronym{awgn}{AWGN}{Additive white Gaussian noise}
\newacronym{fov}{FOV}{field-of-view}
\newacronym{lmm}{LMM}{Linear Mixing Model}
\newacronym{mnf}{MNF}{Maximum Noise Fraction}
\newacronym{mle}{MLE}{Maximum Likelihood Estimation}
\newacronym{tv}{TV}{Total Variation}
\newacronym{tec}{TEC}{thermoelectric cooler}
\newacronym{lcsi}{LCSI}{LCOS Computational Spectral Imager}
\newacronym{swpca}{SWPCA}{Switch Weighted Principal Component Analysis}

%%%%%%%%%%%%%%%%%%% Define symbols here %%%%%%%%%%%%%%%%%%%%%%%%%%%%

\newglossaryentry{specres}{type=symbols, name={$ \delta_{\lambda} $}, symbol={$ \delta_{\lambda} $} ,sort=delta, description={Spectral resolution}}

\newglossaryentry{specchan}{type=symbols, name={$ c $}, symbol={$ c $} ,sort=c, description={Spectral channel index}}


\newglossaryentry{numspecchan}{type=symbols,name={$ N_{\lambda} $}, symbol={$ N_{\lambda} $},sort=Nlambda, description={Number of spectral channels}}

\newglossaryentry{numspeccand}{type=symbols,name={$ N_{R} $}, symbol={$ N_{R} $},sort=Nr, description={Number of spectra in the spectral library}}

\newglossaryentry{measvec}{type=symbols,name={$ \mb{g} $}, symbol={$ \mb{g} $},sort=g, description={Measurement vector}}

\newglossaryentry{objvec}{type=symbols,name={$ \mb{f} $}, symbol={$ \mb{f} $},sort=f, description={Object signal-of-interest}}

\newglossaryentry{estobjvec}{type=symbols,name={$ \mb{ \hat{ f }} $}, symbol={$ \mb{ \hat{ f }} $},sort=fhat, description={Estimated object}}

\newglossaryentry{hadamardn}{type=symbols,name={$ \mb{ H }_N $}, symbol={$ \mb{ H }_N $}, sort=Hn, description={A Hadamard matrix of size $N \times N$}}

\newglossaryentry{H}{type=symbols,name={$ \mb{ H } $}, symbol={$ \mb{ H } $}, sort=H, description={The system matrix which captures all of the physical phenomena in a measurement. Also called the measurement matrix and sensing matrix.}}

\newglossaryentry{noisevec}{type=symbols,name={$ \mb{ e } $}, symbol={$ \bm{ e } $}, sort=e, description={Additive noise vector}}

\newglossaryentry{nummeas}{type=symbols,name={$ N_{m} $}, symbol={$ N_{m} $}, sort=Nm, description={Total number of measurements}}

\newglossaryentry{m}{type=symbols,name={$ m $}, symbol={$ m $}, sort=m, description={measurement number/index}}

\newglossaryentry{n}{type=symbols,name={$ n $}, symbol={$ n $}, sort=n, description={The number of elements in the object \gls{objvec} }}

\newglossaryentry{A}{type=symbols,name={$ \mb{A} $}, symbol={$ \mb{A} $}, sort=A, description={The product of the sensing matrix and the representation matrix $\mb{A} = \mb{H}\mb{\Psi}$ }}

\newglossaryentry{l1}{type=symbols,name={ $\ell_1$ }, symbol={ $\ell_1$ }, sort=l1, description={The $\ell_1$ is a norm which is defined as the sum of the absolute values of the entries a vector}}

\newglossaryentry{tau}{type=symbols,name={ $\tau$ }, symbol={ $\tau$ }, sort=tau, description={Notation for a regularization parameter in an objective function for any kind of optimization}}

\newglossaryentry{sparseRepVec}{type=symbols,name={ $\mb{x}$ }, symbol={ $\mb{x}$ }, sort=x, description={ Notation for a sparse representation vector of the object signal}}

\newglossaryentry{sparseSym}{type=symbols,name={ $K$ }, symbol={ $K$ }, sort=K, description={Notation for sparsity which is defined as the number of non-zero entries in a vector. }}

\newglossaryentry{spectralDataCube}{type=symbols,name={ $D$ }, symbol={ $D$ }, sort=D, description={Notation for the spectral datacube. }}


%%%%%%%%%%%%%%%%%% Define new terms in the glossary %%%%%%%%%%%%%%%%%%%%%%%%%%%%%

\newglossaryentry{isomorphic sensing}
{
name={isomorphic sensing},
text={isomorphic sensing},
description={An isomorphic sensor is any sensor that attempts to produce measurement data that resembles the signal-of-interest. An isomorphic measurement is a measurement that resembles the signal-of-interest. An isomorphic measurement can be described as a one-to-one mapping from the signal-of-interest to the measurement, and is represented in matrix notation with an identity matrix. Isomorphic sensing is synonomous with traditional sensing.}
}

\newglossaryentry{isomorphic}
{
name={isomorphic},
text={isomorphic},
description={\emph{See} \gls{isomorphic sensing}}
}

\newglossaryentry{isomorphic sensor}
{
name={isomorphic sensor},
text={isomorphic sensor},
plural={isomorphic sensors},
description={\emph{See} \gls{isomorphic sensing}}
}

\newglossaryentry{traditional sensing}
{
name={traditional sensing},
text={traditional sensing},
description={\emph{See} \gls{isomorphic sensing}}
}

\newglossaryentry{multiplex sensing}
{
name={multiplex sensing},
description={A multiplexing sensor is any sensor that attempts to combine the physical phenomena of the analog signal-of-interest in to a few or one analog-to-digital sample to overcome limits due to signal-to-noise ratio. The measurement data that does resemble the signal-of-interest. A matrix representation of a multiplex measurement will not look like an identity matrix}
}

\newglossaryentry{multiplexing}
{
name={multiplexing},
text={multiplexing},
description={ \emph{See} \gls{multiplex sensing}}
}

\newglossaryentry{indirect-imaging}
{
name={indirect-imaging},
text={indirect-imaging},
description={An imaging sensor that attempts to reconstruct an image of an object using non-isomorphic measurements. A computational step is required to reconstruct the object signal-of-interest. X-Ray \gls{ct} and \gls{sar} are examples types of indirect-imaging.  }
}

\newglossaryentry{task-specific sensing}
{
name={task-specific sensing},
text={task-specific sensing},
description={A sensor that does not attempt to reconstruct the signal-of-interest to perform a signal processing task such as detection, estimation, and classification. The \gls{afssi-c} is an example of a task-specific sensor. }
}

\newglossaryentry{compressive sensing}
{
name={compressive sensing},
text={compressive sensing},
description={A sensing technique that attempts to directly a compressive or sparse representation of the signal-of-interest during the measurement. Compressive sensors attempt to uses significantly less measurements than the dimensionality of the signal-of-interest. Compressive sensing relies on non-linear optimization algorithms to perform reconstruction or task-specific sensing from highly underdetermined inverse problems. These algorithms often rely on sparsity to avoid overfitting.}
}

\newglossaryentry{sparsity}
{
name={sparsity},
text={sparsity},
description={A set, vector, or matrix which contains an overwhelming majority of zeros relative to the size of the set, vector, or matrix. }
}



\newglossaryentry{compressive sensors}
{
name={compressive sensors},
plural={compressive sensors},
%text={compressive sensors},
description={ \emph{See} \gls{compressive sensing}}
}

\newglossaryentry{compressive sampling}
{
name={compressive sampling},
plural={compressive sampling},
text={compressive sampling},
description={ \emph{See} \gls{compressive sensing}}
}

\newglossaryentry{computational sensing}
{
name={computational sensing},
text={computational sensing},
description={Any sensing technique in which the sensor uses \gls{indirect-imaging}, \gls{multiplex sensing}, \gls{compressive sensing}, or \gls{task-specific sensing}.}
}

\newglossaryentry{computational sensor}
{
name={computational sensor},
text={computational sensor},
plural={computational sensors},
description={ \emph{See} \gls{computational sensing}}
}


\newglossaryentry{compressive imaging}
{
name={compressive imaging},
text={compressive imaging},
description={\gls{compressive sensing} applied to imaging. Typically the goal is to reconstruct the entire object scene. However some compressive imaging sensors are have a task-specific sensing approach, such as the \gls{scout}.}
}

\newglossaryentry{measurement}
{
name={measurement},
text={measurement},
description={A process that converts a physical phenomena to single datum or a set of data. In the context of this dissertation it used synonomously with the detector readout.}
}

\newglossaryentry{monochromator}
{
name={monochromator},
text={monochromator},
plural={monochromator},
description={An optical instrument that transmits a selectable narrow wavelength band of light chosen from a wider range of wavelengths available at the input.}
}

\newglossaryentry{pixel pitch}
{
name={pixel pitch},
text={pixel pitch},
description={The center to center distance between pixels on a focal-plane array such as a CCD or CMOS image sensor.}
}

\newglossaryentry{spectral resolution}
{
name={spectral resolution},
text={spectral resolution},
description={The smallest the smallest difference in wavelength the instrument or sensor can discern.}
}

\newglossaryentry{data processing inequality}
{
name={data processing inequality},
text={data processing inequality},
description={An theorem from information theory that proves the information of a signal cannot be increased via a local physical operation.}
}

\newglossaryentry{coding}
{
name={Coding},
text={coding},
description={In the context of computational sensing, coding is the process of modifying or modulating an analog signal during measurement. Coding is often used to reduce degeneracy in the measurement data. In the context of spectroscopy, the spectral filters act to code the spectrum.  }
}

\newglossaryentry{code}
{
name={code},
text={code},
description={ \emph{See} \gls{coding}}
}

\newglossaryentry{forward model}
{
name={forward model},
text={forward model},
description={A numerical model, typically an equation, that explains the mapping of the analog signal-of-interest to the measurement data.}
}

\newglossaryentry{inverse problem}
{
name={inverse problem},
text={inverse problem},
description={The problem of taking the measurement data and calculating a reconstruction of the signal-of-interst or task-specific parameters. In computational sensing, computer algorithms are used to solve inverse problems.}
}

\newglossaryentry{inversion}
{
name={inversion},
text={inversion},
description={The of solving an \gls{inverse problem}}.
}



\newglossaryentry{sparse}
{
name={sparse},
text={sparse},
description={ \emph{See} \gls{sparsity}}
}

\newglossaryentry{sample}
{
name={sample},
text={sample},
description={ The process of mapping a continuous signal to a discrete signal.  }
}

\newglossaryentry{sampling}
{
name={sampling},
text={sampling},
description={ \emph{See} \gls{sample}  }
}



\newglossaryentry{multiplex advantage}
{
name={multiplex advantage},
text={multiplex advantage},
description={The improvement in \gls{snr} that is due to multiplexed measurements rather than isomorphic measurements. This is often referred to as the Fellgett advantage since it was first discovered by Peter Fellgett. \emph{See} \gls{multiplex sensing}} 
}

\newglossaryentry{Fellgett advantage}
{
name={Fellgett advantage},
text={Fellgett advantage},
description={\emph{See} \gls{multiplex advantage}} 
}

\newglossaryentry{Jacquinot advantage}
{
name={Jacquinot advantage},
text={Jacquinot advantage},
description={This results from the fact that in a dispersive instrument, the traditional spectrometer has entrance and exit slits which restrict the amount of light that passes through it. By removing the slits a spectrometer produces a higher signal-to-noise ratio.} 
}

\newglossaryentry{bandlimited signal}
{
name={bandlimited signal},
text={bandlimited signal},
description={A bandlimited signal is any signal $ g(x) $ that whose Fourier transform $ G(f_x) $ is zero and remains zero past a certain frequency $ \lvert f_x \rvert \geq B$.} 
}

\newglossaryentry{basis pursuit}
{
name={basis pursuit},
text={basis pursuit},
description={Algorithms which attempt to solve a compressive sensing problem by solving the convex optimization problem of 
\begin{equation}
	\mbh{x} = \argminA_{\mb{x}} \: \| \mb{x} \|_{1} \text{\; subject to \;} \mb{A}\mb{x} = \mb{g}
\end{equation} 
}
}

\newglossaryentry{incoherence}
{
name={incoherence},
text={incoherence},
description={ The idea that }
}

\newglossaryentry{compressible}
{
name={compressible},
text={compressible},
description={ Signals with strictly sparse representation vectors are unlikely. Fortunately, it is possible to have approximately sparse representation vectors, which are called compressible. In other words, the sorted magnitudes of the coefficients $|x_n|$ quickly decay.}
}

\newglossaryentry{representation basis}
{
name={representation basis},
text={representation basis},
description={ The idea that representation basis}
}

\newglossaryentry{posterior}
{
name={posterior},
text={posterior},
description={ The posterior probability is the conditional probability of a hypothesis $h$ or parameter $\theta$ given some data $g$. $ P \left( \theta \given[\big] g \right) $}
}

\newglossaryentry{prior}
{
name={prior},
text={prior},
description={ The priori probability is the probability of a hypothesis $h$ or parameter $\theta$ without any knowledge of the data data $g$. $ P \left( \theta \right) $}
}

\newglossaryentry{likelihood}
{
name={likelihood},
text={likelihood},
description={ The likelihood is the probability of the data $g$ assuming that a hypothesis $h$ or parameter $\theta$ is true. $P \left( g \given[\big] \theta \right)$}
}

\newglossaryentry{Maximum A Posteriori}
{
name={Maximum A Posteriori (MAP)},
text={Maximum A Posteriori (MAP)},
description={Maximum A Posteriori (MAP) estimator says the parameters  $ \mb{\theta} $ which maximize the \gls{posterior} probability are the most likely parameters.}
}

\newglossaryentry{lasso}
{
name={lasso},
text={lasso},
description={The least absolute shrinkage and selection operator (lasso) is a regression analysis method that performs both variable selection and regularization. It is refers to an optimization problem, a regression technique, and an algorithm. The lasso problem is the
%
\begin{equation*}
	\mbh{x} = \argminA_{\mb{x}} \: \| \mb{Ax} = \mb{g} \|_{2}^{2} + \tau \| \mb{x} \|_1
\end{equation*}
 }
}

\newglossaryentry{ridge regression}
{
name={ridge regression},
text={ridge regression},
description={Ridge regression is a regression technique that uses $\ell_2$ regularization to prevent overfitting of the data. Ridge regression seeks coefficient estimates that fit the data well by making the objective function, the $\ell_2$ norm between the data and the linear model small, however the shrinkage penalty has the effect of shrinking estimates towards zero. 
%
\begin{equation*}
	\mbh{x} = \argminA_{\mb{x}} \: \| \mb{Ax} = \mb{g} \|_{2}^{2} + \tau \| \mb{x} \|_2
\end{equation*}
 }
}

\newglossaryentry{whiskbroom}
{
name={Whiskbroom},
text={whiskbroom},
description={Whiskbroom scanning is an isomorphic measurement technique for acquiring a spectral datacube. In the whiskbroom technique, the entire spectrum at each spatial location is acquired one at a time.}
}

\newglossaryentry{pushbroom}
{
name={Pushbroom},
text={pushbroom},
description={Pushbroom scanning is an isomorphic measurement technique for acquiring a spectral datacube. In the pushbroom technique, the entire spectrum for an entire spatial row or column is acquired one at a time.}
}


\newglossaryentry{tunable filter}
{
name={Tunable filter},
text={tunable filter},
description={Pushbroom scanning is an isomorphic measurement technique for acquiring a spectral datacube. In the pushbroom technique, the entire spectrum for an entire spatial row or column is acquired one at a time.}
}

\newglossaryentry{mixed spectrum}
{
name={Mixed spectrum},
text={mixed spectrum},
description={A mixed spectrum is a measured spectrum that contains spectra from more than one spectrum.}
}

\newglossaryentry{dimensionality reduction}
{
name={dimensionality reduction},
text={dimensionality reduction},
description={The process of reducing the dimensionality of the data in the scene. This step is optional and is only invoked by some algorithms to reduce the computional load of subsequent steps. }
}

\newglossaryentry{endmember determination}
{
name={endmember determination},
text={endmember determination},
description={The process of reducing the which endmembers are present in a mixed spectrum. }
}

\newglossaryentry{fractional abundance}
{
name={fractional abundance},
text={fractional abundance},
description={The relative amount of a spectral endmember in a mixed spectrum}
}

\newglossaryentry{endmember}
{
name={endmember},
text={endmember},
plural={endmembers},
description={The constituent spectra in a mixed spectrum is called the endmember}
}

\newglossaryentry{spectral unmixing}
{
name={spectral unmixing},
text={spectral unmixing},
description={The procedure by which the measured spectrum of a mixed pixel is decomposed into a collection of constituent spectra, or \glspl{endmember}, and a set of corresponding \glspl{fractional abundance}.}
}


